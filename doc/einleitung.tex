\section{Einleitung}
Wasser verlässt ein Gebiet in dem es entweder als Oberflächenabfluss, Basisabfluss oder Grundwasser abfliesst, oder es verdunstet. Wird das Wasser von der Bodenoberfläche verdunstet, spricht man von Evaporation. Verdunstet es von Pflanzenoberflächen, so ist dies die Transpiration. Da es schwierig ist, diese beiden Messgrössen technisch voneinander zu trennen, werden sie zu einer Grösse, der Evapotranspiration, zusammengefasst.

In diesem Versuch geht es nun darum, die Evapotranspiration mit verschiedenen Modellen zu bestimmen. Für die Hydrologie ist es wichtig, die Evapotranspiration zu kennen, da sie ein wichtiger Bestandteil in der Wasserbilanz eines Einzugsgebiets ist.

Es gibt verschiedene Methoden, die Evapotranspiration zu bestimmen. Zum einen gibt es empirische Modelle, zum anderen kann die Evapotranspiration auch mit Lysimetern gemessen werden. Empirische Modelle werden entwickelt, in dem die Evapotranspiration als Funktion von meteorologischen Variablen beschrieben wird. Es handelt sich dabei um meteorologische Variablen, die einen starken Einfluss auf die Evaporation haben, zum Beispiel Lufttemperatur, globale Strahlung, Sonnenscheindauer, Luftfeuchtigkeit, etc. Dazu wird die Evapotranspiration in einem Versuchsgebiet gemessen und mit den Meteodaten in Verbindung gesetzt. Da diese Modelle für bestimmte Standorte mit zugehörigen Eigenschaften entwickelt werden, sind sie nicht immer auf andere Standorte übertragbar und können nicht als allgemein gültig angenommen werden.

Für die Auswertungen stehen Lysimeterdaten von der Forschungsanstalt Agroscope Reckenholz-Tänikon\,ART und Meteodaten der ANETZ Station in Reckenholz zur Verfügung. Mit den Lysimeter- und den Meteodaten kann die reale Evapotranspiration direkt berechnet werden. Für die Modellierung der potentiellen Evapotranspiration mit der FAO\,Penman-Monteith-, Turc- und Ivanov-Methode werden nur die Meteodaten verwendet.

Ziel des Versuchs ist es, zu verstehen, was die Evapotranspiration ist und welche Methoden es gibt, diese zu bestimmen. Dazu wird der Zusammenhang zwischen der gemessenen Evapotranspiration und den entsprechenden meteorologischen Grössen diskutiert. Zusätzlich werden die Resultate der obigen Methoden zur Bestimmung der potentiellen Evapotranspiration miteinander verglichen. Ein besonderes Augenmerk soll dabei auf das unterschiedliche Verhalten der einzelnen empirischen Methoden bei unterschiedlicher zeitlicher Auflösung gerichtet werden.