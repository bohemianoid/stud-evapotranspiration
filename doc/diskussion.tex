\section{Diskussion}




\subsection{Sensitivitätsanalyse}
Die Sensitivitätsanalyse der Penman-Monteith Methode zeigt, dass alle Parameter einen etwa gleich grossen Einfluss auf die potentielle Evapotranspiration haben. Da die Methode auf empirische Beobachtungen basiert, kann angenommen werden, dass Faktoren, die einen nur geringen Einfluss auf die Evapotranspiration haben, in der Formel vernachlässigt oder in die numerischen Konstanten einbezogen wurden.

Bei der Methode nach Turc ist bereits aus der Formel (\ref{eq:turc}) ersichtlich, dass die relative Luftfeuchtigkeit linear in die Berechnung der Evapotranspiration eingeht und somit auch den Grössten Einfluss auf das Resultat hat. Die anderen Grössen fliessen nicht linear ein, was auch die Sensitivitätsanalyse widerspiegelt.

Die Methode nacht Ivanov zeigt bei tiefen Temperaturen eine ausgeglichene Sensitivität auf die beiden Parameter. Für höhere Temperaturen reagiert die Evapotranspiration allerdings sehr viel sensitiver auf die Lufttemperaturen. Daher ist diese Methode auch nur für tiefere Temperaturen in den Monaten November bis Februar anwendbar.