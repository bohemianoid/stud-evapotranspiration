\section{Diskussion}
Die reale Evapotranspiration zeigt grosse Unregelmässigkeiten und beinhaltet viele Ausreisser. Dies ist damit zu erklären, dass durch die sehr empfindlichen Waagen der fallende Niederschlag als zu grosse Speicheränderung aufgefasst wird. Sobald der Niederschlag aufhört, wird fälschlicherweise eine zu hohe Evapotranspiration aufgezeichnet.

\subsection{Korrelation zwischen realer Evapotranspiration und meteorologischen Variablen}
Die Globalstrahlung zeigt die grösste Korrelation. Diese überträgt einen Teil der Energie der Sonne auf die Erde. Die Evapotranspiration wird durch den Energieaustausch auf der Oberfläche der Vegetation gesteuert und ist darum durch die zur Verfügung stehende Menge an Energie limitiert. Die Energie ist in Form sensibler Wärme oder Strahlungsenergie präsent (\cite{fao} S.11). Die Energie, die durch die Strahlung zur Verfügung steht, ist dabei die wichtigste Energiequelle. Es erstaunt daher nicht, dass die Globalstrahlung die grösste Korrelation mit der realen Evapotranspiration aufweist.

Die sensible Wärme wird durch die Temperatur dargestellt. Zwischen Temperatur und realer Evapotranspiration ist eine nicht zu vernachlässigende Korrelation vorhanden. Der kleinere Korrelationskoeffizient verdeutlicht dabei die untergeordnete Rolle der sensiblen Wärme bei der Beeinflussung der Evapotranspiration.

Die Evapotranspirationsrate ist umso geringer, je grösser die relative Luftfeuchtigkeit ist. Dies zeigt sich im negativen Vorzeichen des Korrelationskoeffizienten. Der Wert des Korrelationskoeffizienten bewegt sich im gleichen Bereich wie bei der Temperatur. Somit kann gezeigt werden, dass das Feuchtedefizit der Luft ebenfalls ein wichtiger Einflussfaktor für die Bestimmung der Evapotranspiration ist.

Die Windgeschwindigkeit hingegen zeigt keine signifikante Korrelation mit der Evapotranspiration. In der Schweiz erreicht die relative Luftfeuchtigkeit nicht so kritische Werte, die die Evapotranspiration stark beeinflussen würde. Aus diesem Grund ist es für die Pflanze nicht wichtig, dass die feuchte Luft stetig wegtransportiert wird. Somit ist die Evapotranspiration im mitteleuropäischen Klima nur gering vom Wind abhängig.
Die meteorologischen Parameter korrelieren besser mit den Monatsmittelwerten der realen Evapotranspiration als mit den Tagesmittelwerten, weil Extremwerte und Ausreisser wegfallen und allgemein Schwankungen ausgeglichen werden. 

\subsection{Vergleich der verschiedenen Methoden zur Berechnung der potentiellen Evapotranspiration PET}

Die potentielle Evapotranspiration quantifiziert die theoretisch maximal mögliche Evapotranspirationsrate ohne Limitierung durch die Wasserverfügbarkeit. Aus dieser Definition folgt, dass die reale Evapotranspiration die potentielle Evapotranspiration nicht überschreiten darf. Die Resultate zeigen allerdings, dass die AET in den Frühlingsmonaten und im Frühsommer die PET übertrifft. Dies lässt sich dadurch erklären, dass die untersuchten empirischen Methoden nicht spezifisch für die Schweiz entwickelt wurden. Zudem beinhalten Modelle Unsicherheiten und lassen sich nicht vollständig auf die Realität anwenden.

Die gedämpfte Reaktion der Penman-Monteith Methode auf die meteorologischen Grössen beruht auf der Komplexität der Formel und der grossen Anzahl an Inputparametern. Diese wirken sich teilweise gegenläufig auf die potentielle Evapotranspiration aus. Die Methode nach Turc und Ivanov hat weniger Inputparameter und diese zeigen einen direkteren Einfluss auf die PET.

Die Wachstumsphasen der Pflanzen widerspiegeln sich im Anstieg der Evapotranspirationsrate. In diesem Versuch wurde ein konstanter Pflanzenfaktor angenommen. Deshalb stimmt die Wachstumsphase aus der PET nicht mit jener aus der realen Evapotranspiration überein. Der relative Verlauf der realen sowie der beiden potentiellen Evapotranspirationsraten stimmt überein.

\subsection{Vergleich Raps und Weizen}
Aus dem Vergleich von Raps und Weizen folgt, dass jede Pflanze einen charakteristischen Verlauf der jährlichen Evapotranspirationskurve aufweist. Daraus lassen sich die unterschiedlichen Wachstumsphasen erkennen.

\subsection{Sensitivitätsanalyse}
Die Sensitivitätsanalyse der Penman-Monteith Methode zeigt, dass alle Parameter einen etwa gleich grossen Einfluss auf die potentielle Evapotranspiration haben. Da die Methode auf empirischen Beobachtungen basiert, kann angenommen werden, dass Faktoren, die einen nur geringen Einfluss auf die Evapotranspiration haben, in der Formel vernachlässigt oder in die numerischen Konstanten einbezogen wurden.

Bei der Methode nach Turc ist bereits aus der Formel (\ref{eq:turc}) ersichtlich, dass die relative Luftfeuchtigkeit linear in die Berechnung der Evapotranspiration eingeht und somit auch den Grössten Einfluss auf das Resultat hat. Die anderen Grössen fliessen nicht linear ein, was auch die Sensitivitätsanalyse widerspiegelt.

Die Methode nach Ivanov zeigt bei tiefen Temperaturen eine ausgeglichene Sensitivität auf die beiden Parameter. Für höhere Temperaturen reagiert die Evapotranspiration allerdings sehr viel sensitiver auf die Lufttemperaturen. Daher ist diese Methode auch nur für tiefere Temperaturen in den Monaten November bis Februar anwendbar.