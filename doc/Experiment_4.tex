\documentclass[12pt,a4paper,tablecaptionabove]{scrartcl} 
\usepackage{ngerman}   % neue deutsche Rechtschreibung (Trennregeln ...)
\usepackage[utf8]{inputenc} % UTF8 Zeichenkodierung, falls das nicht
                            % funktioniert
                            % \usepackage[latin1]{inputenc} verwenden
\usepackage[T1]{fontenc} % Deutsche Umlaute und so
\usepackage{ae}        % Schriften für Adobe Acrobat
\usepackage{textcomp}  % mehr Sonderzeichen
\usepackage{calc}      % Paket für die Berechnung von Längen
\usepackage{color}     % Farbiger Text etc.
\usepackage{graphicx}  % importieren von Graphiken
\usepackage{subfigure} % Mehrere Abbildungen in einer figur
\usepackage{booktabs}  % für schöne Tabellen
\usepackage[numbers]{natbib}    % Für Literaturverzeichnis nach DIN 1505
\usepackage{scrpage2}  % Kopf- und Fusszeilen
\usepackage{tabularx}  % Tabellen auf Textbreite einpassen
\usepackage{verbatim}  % Programmcode angeben
\usepackage{listings}  % Programmcode mit mehr Optionen
\usepackage{hyperref}  % Links innerhalb des PDF-Files
\usepackage{subscript} % Hoch-, Tiefstellen
\usepackage{float} % das die cheibe Grafike au döt bliibe, wo si sötte
\usepackage{pdflscape} % Hoch-, Querformat
\usepackage{a4wide} % Textbreite
\usepackage{pstricks, pst-node, pst-plot, pst-circ}
\usepackage{moredefs}




% Titel definieren
\title{Messung und Modellierung der Evapotranspiration}
\author{Debora Jäckel\\Simon Roth\\Gabriela Schär \\Alexandra Schuler}
\date{\today{}}

\begin{document}


% Schriften für Abbildungsbeschriftungen
\renewcommand*{\capfont}{\normalfont}
\renewcommand*{\caplabelfont}{\sffamily\bfseries}


% Kopf- und Fusszeilen definieren
\pagestyle{scrheadings}

% Dynamische Kopfzeilen
%\automark[section]{chapter}

% Statische Kopfzeilen
\ihead{Labor II}
\chead{ }
\ohead{Experiment 4}
\ifoot{}
\cfoot{ }
\ofoot{\pagemark}

% Trennstriche zwischen Kopf- und Fusszeile
\setheadsepline{0.5pt}
\setfootsepline{0.5pt}

% Schrftart für Kopfzeile
\setkomafont{pagehead}{\normalfont\slshape}

% Titel ausgeben
\maketitle

\newpage

\begin{abstract}

blablabla...

\end{abstract}

\newpage
% Inhaltsverzeichnis
\tableofcontents
% Neue Seite
\newpage


\section{Einleitung}

Wasser verlässt ein Gebiet in dem es entweder als Oberflächenabfluss, Basisabfluss oder Grundwasser abfliesst, oder es Verdunstet. Wird das Wasser von der Bodenoberfläche verdunstet, spricht man von Evaporation. Verdunstet es von Pflanzenoberflächen, so ist dies die Transpiration. Da es schwierig ist, diese beiden Messgrössen technisch voneinander zu trennen, werden sie zu einer Grösse, der Evapotranspiration, zusammengefasst.

In diesem Versuch geht es nun darum, Evapotranspiration mit verschiedenen Modellen zu bestimmen. Für die Hydrologie ist es wichtig, die Evapotranspiration zu kennen, da sie ein wichtiger Bestandteil in der Wasserbilanz eines Einzugsgebiets ausmacht.

Es gibt verschiedene Methoden, die Evapotranspiration zu bestimmen. Zum einen gibt es empirische Modelle, zum anderen kann die Evapotranspiration auch mit Lysimetern gemessen werden. Empirische Modelle werden entwickelt, in dem die Evapotranspiration als Funktion von meteorologischen Variablen beschrieben wird. Es handelt sich dabei um meteorologische Variablen, die einen starken Einfluss auf die Evaporation haben, zum Beispiel Lufttemperatur, globale Strahlung, Sonnenscheindauer, Luftfeuchtigkeit, etc. Dazu wird die Evapotranspiration in einem Versuchsgebiet gemessen und mit den Meteodaten in Verbindung gesetzt. Da diese Modelle für bestimmte Standorte mit zugehörigen Eigenschaften entwickelt werden, sind sie nicht immer auf andere Standorte übertragbar und können nicht als allgemein gültig angenommen werden.

Für die Auswertungen stehen Lysimeterdaten von der Forschungsanstalt Agroscope Reckenholz-Tänikon ART und Meteodaten der ANETZ Station in Reckenholz zur Verfügung. Mit den Lysimeter- und den Meteodaten kann die Evapotranspiration direkt berechnet werden. Für die Modellierung mit der FAO Penman-Monteith-, Turc- und Ivanov-Methode werden nur die Meteodaten verwendet.

Ziel des Versuchs ist es, zu verstehen, was die Evapotranspiration ist und welche Methoden es gibt, diese zu bestimmen.
\section{Methode}

Für die Bestimmung der Evapotranspiration werden die Lysimeter- und Meteodaten mit MATLAB prozessiert. Da die Meteodaten in UTC und die Lysimeterdaten in MEZ vorliegen, müssen die Datensätze zuerst an eine Zeit angepasst werden. Zudem müssen die Einheiten so angepasst werden, dass sie auf die jeweiligen Berechnungsformeln passen.

\subsection{FAO Penman-Monteith}

Die Penman-Monteith-Methode kombiniert die beiden Ansätze der Massenbilanz und der Energiebilanz. Alle in der Formel enthaltenen Parameter können entweder direkt gemessen oder dann aus meteorologischen Daten berechnet werden. Die Methode impliziert, dass der aerodynamische und der Oberflächenwiderstand von der Oberflächenbepflanzung abhängig sind [vgl.\,Abb.\,\ref{fig:widerstand}]. 

\begin{figure}[H]
\centering
\includegraphics[width=0.8\textwidth]{figures/penman_widerstand.jpg}
\caption{Schematische Darstellung des aerodynamischen und des Oberflächenwiderstands in der Penman-Monteith-Methode}
\label{fig:widerstand}
\end{figure}

Der aerodynamische Widerstand beschreibt die Grösse, welche Wärme und Wasserdampf daran hindert , wegtransportiert zu werden. Diese hängt von der Windgeschwindigkeit und der Bodenrauigkeit ab. Der Oberflächenwiderstand beschreibt den Widerstand des Wasserdampfes, sich zwischen den transpirierenden Pflanzen und dem evaporierendem Boden zu bewegen. Dieser ist abhängig von Verhältnis der Blattfläche zur Bodenfläche und dem Stomatawiderstand eines gut bestrahlten Blattes. Unter Berücksichtigung dieser Einflüsse folgt die FAO\,Penman-Monteith-Formel für die Referenzfäche:

\begin{equation}
\label{eq:penman_ref}
ET_0=\frac{0.408\Delta \left(R_n-G\right)+\gamma \frac{900}{T+273}u_2\left(e_s-e_a\right)}{\Delta +\gamma\left(1+0.34u_2\right)}
\end{equation}
\begin{table}[H]
\centering
\begin{tabular}{ll}
$\mathrm{ET_0}$ & potentielle Referenzevapotranspiration $\mathrm{[mm/s]}$\\
$\mathrm{R_n}$ & Nettostrahlung $\mathrm{[W/m^2]}$ \\
$\mathrm{G}$ & Bodenwärmefluss $\mathrm{[W/m^2]}$\\
$\mathrm{\Delta}$ & Steigung der Sättigungsdampfdruckkurve $\mathrm{[kPa/^{\circ}C]}$\\
$\mathrm{\gamma}$ & Psychrometerkonstante $\mathrm{[kPa/^{\circ}C]}$\\
$\mathrm{T}$ & mittlere Temperatur in 2\,m Höhe $\mathrm{[^{\circ}C]}$\\
$\mathrm{u_2}$ & Windgeschwindigkeit in 2\,m Höhe $\mathrm{[m/s]}$\\
$\mathrm{e_s}$ & Sättigungsdampfdruck $\mathrm{[kPa]}$\\
$\mathrm{e_a}$ & aktueller Dampfdruck [kPa]\\
\end{tabular}
\end{table}

In den folgenden Berechnungen wird der Bodenwärmefluss allerdings vernachlässigt. Um die Evapotranspiration einer spezifischen Pflanze zu bestimmen, wird die Referenzevapotranspiration mit einem Pflanzenfaktor multipliziert. Es folgt daraus:

\begin{equation}
\label{eq:penman_spez}
ET_C=K_C*ET_0
\end{equation}
\begin{table}[H]
\centering
\begin{tabular}{ll}
$\mathrm{ET_0}$ & potentielle Referenzevapotranspiration $\mathrm{[mm/s]}$\\
$\mathrm{K_C}$ & Pflanzenfaktor [-]\\
$\mathrm{ET_C}$ & Evapotranspiration einer spezifischen Pflanze $\mathrm{[mm/s]}$\\\
\end{tabular}
\end{table}

Die Nettostrahlung $\mathrm{R_n}$ und die Temperatur T können aus den Meteodaten entnommen werden. Die übrigen Parameter müssen berechnet werden. Die FAO\,\cite{fao} gibt folgende Formeln:

\begin{description}
\item[Steigung der Sättigungsdampfdruckkurve]~
\begin{equation}
\label{eq:steigung}
\Delta=\frac{4098\left[0.6108*e^{\frac{17.27*T}{T+237.3}}\right]}{\left(T+237.3\right)^2}
\end{equation}
\begin{table}[H]
\centering
\begin{tabular}{ll}
$\mathrm{\Delta}$ & Steigung der Sättigungsdampfdruckkurve $\mathrm{[kPa/^{\circ}C]}$\\
T & Temperatur mittlere Temperatur in 2\,m Höhe $\mathrm{[^{\circ}C]}$\\
\end{tabular}
\end{table}


\end{description}


\section{Resultate}




\subsection{Sensitivitätsanalyse}
Die Penman-Montheit Methode zeigt die grösste Sensitivität gegenüber des Sättigungsdampfdruckes. Die Veränderungen von Niederschlagsmenge, relativer Luftfeuchtigkeit und Sonnenscheindauer zeigen kaum eine Änderung der Evapotranspiration. Die Methode nach Turc zeigt die grösste Sensitivität gegenüber der relativen Luftfeuchtigkeit und die Lufttemperatur hat den kleinsten Einfluss. Die Methode nach Ivanov reagiert auf beide Parameter Lufttemperatur und relative Luftfeuchtigkeit etwa gleich empfindlich. Die genauen Resultate und die Berechnung ist im Anhang\,\ref{sec:sensitivitaet} ersichtlich.
\section{Diskussion}




\subsection{Sensitivitätsanalyse}
Die Sensitivitätsanalyse der Penman-Monteith Methode zeigt, dass alle Parameter einen etwa gleich grossen Einfluss auf die potentielle Evapotranspiration haben. Da die Methode auf empirische Beobachtungen basiert, kann angenommen werden, dass Faktoren, die einen nur geringen Einfluss auf die Evapotranspiration haben, in der Formel vernachlässigt oder in die numerischen Konstanten einbezogen wurden.

Bei der Methode nach Turc ist bereits aus der Formel (\ref{eq:turc}) ersichtlich, dass die relative Luftfeuchtigkeit linear in die Berechnung der Evapotranspiration eingeht und somit auch den Grössten Einfluss auf das Resultat hat. Die anderen Grössen fliessen nicht linear ein, was auch die Sensitivitätsanalyse widerspiegelt.

Die Methode nacht Ivanov zeigt bei tiefen Temperaturen eine ausgeglichene Sensitivität auf die beiden Parameter. Für höhere Temperaturen reagiert die Evapotranspiration allerdings sehr viel sensitiver auf die Lufttemperaturen. Daher ist diese Methode auch nur für tiefere Temperaturen in den Monaten November bis Februar anwendbar.
\section{Schlussfolgerung}

Der Versuch hat gezeigt, dass die Evapotranspiration ein sehr komplexer Vorgang ist. Viele verschiedene Umweltfaktoren beeinflussen die Evapotranspirationsrate. Diese Faktoren lassen sich oft nicht einfach bestimmen und ihr Einfluss ist nicht immer direkt ersichtlich. Um die Evapotranspiration modellieren zu können, wurden viele verschiedene empirische Modelle entwickelt und diese sind meist nur für eine klimatische Region gültig. Will man die Evapotranspirationsrate modellieren ist es wichtig, dass man jene Methode auswählt, welche am besten auf die Fragestellung und die geografischen Gegebenheiten passt. So werden zum Beispiel bei der Penman-Monteith Methode zahlreiche meteorologische Grössen berücksichtigt, allerdings wird durch diese Komplexität die direkte Abhängigkeit von den meteorologischen Bedingungen abgeschwächt.
In diesem Versuch wurde ein über das ganze Jahr hinweg konstanter Pflanzenfaktor verwendet. Dieser kann allerdings der Wachstumsphase der jeweiligen Pflanze angepasst werden. Speziell für die Landwirtschaft in ariden Gebieten ist es nützlich, über die Evapotranspiration der Pflanzen Bescheid zu wissen. So können Aussaat, Ernte und Bewässerung dem Klima angepasst werden und die Ressourcen und den Ertrag optimiert werden. 


\begin{appendix}

\section{Penman-Monteith}
\label{sec:penman}

\begin{description}
\item[Steigung der Sättigungsdampfdruckkurve]
\begin{equation}
\label{eq:delta}
\Delta=\frac{4098\left[0.6108*e^{\frac{17.27*T}{T+237.3}}\right]}{\left(T+237.3\right)^2}
\end{equation}
\begin{table}[H]
\centering
\begin{tabular}{ll}
$\mathrm{\Delta}$ & Steigung der Sättigungsdampfdruckkurve $\mathrm{[kPa/^{\circ}C]}$\\
T & mittlere Temperatur in 2\,m Höhe $\mathrm{[^{\circ}C]}$\\
\end{tabular}
\end{table}

\item[Nettostrahlung]

\begin{figure}[H]
\centering
\includegraphics[width=0.8\textwidth]{figures/strahlung.jpg}
\caption{Schematische Darstellung der unterschiedlichen Strahlungsarten. $\mathrm{R_{s}}$ ist die kurzwellige Strahlung, $\mathrm{R_{l}}$ die langwellige Strahlung und $\mathrm{R_{a}}$ die atmosphärische Strahlung (aus \cite{fao})}
\label{fig:strahlung}
\end{figure}

\begin{equation}
\label{eq:rn}
R_n=0.77*R_s-\sigma\left[\frac{T_{max,K}^4+T_{min,K}^4}{2}\right]\left(0.34-0.14\sqrt{e_a}\right)\left(1.35\frac{R_s}{R_{s0}}-0.35\right)
\end{equation}
\begin{table}[H]
\centering
\begin{tabular}{ll}
$\mathrm{R_n}$ & Nettostrahlung $\mathrm{[MJ/m^2d]}$ \\
$\mathrm{R_s}$ & Kurzwellenstrahlung $\mathrm{[MJ/m^2d]}$ \\
$\mathrm{\sigma}$ & Stefan Boltzmann Konstante $\mathrm{[4.903*10^{-9}\,MJ/K^4m^2d]}$\\
$\mathrm{T_{max,K}}$ & maximale Temperatur während 24 h [K]\\
$\mathrm{T_{min,K}}$ & minimale Temperatur während 24 h [K]\\
$\mathrm{e_a}$ & aktueller Dampfdruck [kPa]\\
$\mathrm{R_{s0}}$ & Kurzwellenstrahlung ohne Wolkenbedeckung $\mathrm{[MJ/m^2d]}$\\
\end{tabular}
\end{table}

\begin{equation}
\label{eq:rs0}
R_{s0}=\left(0.75+2*10^{-5}z\right)*R_a
\end{equation}
\begin{table}[H]
\centering
\begin{tabular}{ll}
$\mathrm{R_s0}$ & Kurzwellenstrahlung ohne Wolkenbedeckung $\mathrm{[MJ/m^2d]}$\\
$\mathrm{R_a}$ & extraterrestrische Strahlung $\mathrm{[MJ/m^2d]}$ \\
z & Höhe über Meer [m]\\
\end{tabular}
\end{table}

\begin{equation}
\label{eq:Ra_short_period}
R_{a}=\frac{12 (60)}{\pi}G_{sc}*d_{r}[(\omega _{2}-\omega _{1})sin(\varphi)sin(\delta)+cos(\varphi)cos(\delta)(sin(\omega _{2})- sin(\omega _{1}))]
\end{equation}
\begin{table}[H]
\centering
\begin{tabular}{ll}
R$\mathrm{_{a}}$ & extraterrestrische Strahlung in einer Stunde (oder in kürzerem Zeitintervall) [MJ/m$\mathrm{^{2}}$h]\\
G$\mathrm{_{sc}}$ & Solarkonstante = 0.0820 MJ/m$\mathrm{^{2}}$min\\
d$\mathrm{_{r}}$ & inverse relative Distanz Sonne-Erde\\
$\mathrm{\delta}$ & solare Deklination [rad]\\
$\mathrm{\varphi}$ & geografische Breite [rad]\\
$\mathrm{\omega_{1}}$ & Sonneneinstrahlwinkel am Anfang der Zeitperiode [rad]\\
$\mathrm{\omega_{2}}$ & Sonneneinstrahlwinkel am Ende der Zeitperiode [rad]\\
\end{tabular}
\end{table}

\begin{equation}
\label{eq:Ra_long_period}
R_{a}=\frac{24 (60)}{\pi}G_{sc}*d_{r}[\omega _{s}sin(\varphi)sin(\delta)+cos(\varphi)cos(\delta)sin(\omega _{s})]
\end{equation}
\begin{table}[H]
\centering
\begin{tabular}{ll}
R$\mathrm{_{a}}$ & extraterrestrische Strahlung in einem Tag (oder in längerem Zeitintervall) [MJ/m$\mathrm{^{2}}$d]\\
G$\mathrm{_{sc}}$ & Solarkonstante = 0.0820 MJ/m$\mathrm{^{2}}$min\\
d$\mathrm{_{r}}$ & inverse relative Distanz Sonne-Erde [m]\\
$\mathrm{\delta}$ & solare Deklination [rad]\\
$\mathrm{\varphi}$ & geografische Breite [rad]\\
$\mathrm{\omega_{s}}$ & Sonneneinstrahlwinkel [rad]\\
\end{tabular}
\end{table}

\begin{equation}
\label{eq:dr}
d_r=1+0.033cos\left(\frac{2\pi}{365}J\right)
\end{equation}
\begin{equation}
\label{eq:delta_radiation}
\delta=0.409sin\left(\frac{2\pi}{365}J-1.30\right)
\end{equation}
\begin{table}[H]
\centering
\begin{tabular}{ll}
$\mathrm{d_r}$ & inverse Distanz Sonne-Erde [m]\\
$\mathrm{\delta}$ & solare Deklination [rad]\\
J & Korrekturfaktor (siehe \cite{fao} Annex 2 Table 2.5)\\
\end{tabular}
\end{table}

\begin{equation}
\label{eq:omega_s}
\omega_s=arccos[-tan(\varphi)tan(\delta)]
\end{equation}
\begin{table}[H]
\centering
\begin{tabular}{ll}
$\mathrm{\omega_s}$ & Sonneneinstrahlwinkel [rad]\\
$\mathrm{\varphi}$ & geografische Breite [rad]\\
$\mathrm{\delta}$ & solare Deklination [rad]\\
\end{tabular}
\end{table}

\begin{equation}
\label{eq:omega_i}
\omega_1=\omega-\frac{\pi t_i}{24}
\end{equation}
\begin{equation}
\omega_2=\omega+\frac{\pi t_i}{24}
\end{equation}
\begin{table}[H]
\centering
\begin{tabular}{ll}
$\mathrm{\omega}$ & Sonneneinstrahlwinkel [rad]\\
$\mathrm{t_i}$ & Zeitintervaldauer [h]\\
\end{tabular}
\end{table}

\begin{equation}
\label{eq:omega}
\omega=\frac{\pi}{12}[(t+0.06667(L_z-L_m)+S_c)-12]
\end{equation}
\begin{table}[H]
\centering
\begin{tabular}{ll}
$\mathrm{\omega}$ & Sonneneinstrahlwinkel [rad]\\
t & Zeit [h]\\
$\mathrm{L_z}$ & Längengrad in der Mitte der Zeitzone [rad]\\
$\mathrm{L_m}$ & Längengrad des Messpunktes [rad]\\
$\mathrm{S_c}$ & saisonaler Korrekturfaktor [h]\\
\end{tabular}
\end{table}

\begin{equation}
\label{eq:s_c}
S_c=0.1645sin(2b)-0.1255cos(b)-0.025sin(b)
\end{equation}
\begin{equation}
b=\frac{2\pi(J-81)}{364}
\end{equation}
\begin{table}[H]
\centering
\begin{tabular}{ll}
$\mathrm{S_c}$ & saisonaler Korrekturfaktor [h]\\
J & Korrekturfaktor (siehe \cite{fao} Annex 2 Table 2.5)\\
\end{tabular}
\end{table}

\item[Psychrometerkonstante]
\begin{equation}
\label{eq:gamma}
\gamma=0.665*10^{-3} P
\end{equation}
\begin{table}[H]
\centering
\begin{tabular}{ll}
$\mathrm{\gamma}$ & Psychrometerkonstante $\mathrm{[kPa/^{\circ}C]}$\\
P & Atmosphärendruck [kPa]\\
\end{tabular}
\end{table}

\item[Windgeschwindigkeit in 2 m Höhe]
\begin{equation}
\label{eq:u2}
u_2=u_z\frac{4.87}{ln(67.8z-5.42)}
\end{equation}
\begin{table}[H]
\centering
\begin{tabular}{ll}
$\mathrm{u_2}$ & Windgeschwindigkeit in 2\,m Höhe $\mathrm{[m/s]}$\\
$\mathrm{u_z}$ & Windgeschwindigkeit in z\,m Höhe $\mathrm{[m/s]}$\\
z & Messhöhe [m]\\
\end{tabular}
\end{table}



\item[Sättigungsdampfdruck]
\begin{equation}
\label{eq:es}
e_s=\frac{e^{\circ}(T_{max})+e^{\circ}(T_{min})}{2}
\end{equation}
\begin{table}[H]
\centering
\begin{tabular}{ll}
$\mathrm{e_s}$ & mittlerer Sättigungsdampfdruck $\mathrm{[kPa]}$\\
$\mathrm{e^{\circ}}$ & Sättigungsdampfdruck bei Temperatur T $\mathrm{[kPa]}$\\
T & mittlere Temperatur in 2\,m Höhe $\mathrm{[^{\circ}C]}$\\
\end{tabular}
\end{table}

\begin{equation}
\label{eq:enull}
e^{\circ}(T)=0.6108*e^{\frac{17.27T}{T+273.3}}
\end{equation}
\begin{table}[H]
\centering
\begin{tabular}{ll}
$\mathrm{e^{\circ}(T)}$ & Sättigungsdampfdruck bei Temperatur T $\mathrm{[kPa]}$\\
T & mittlere Temperatur in 2\,m Höhe $\mathrm{[^{\circ}C]}$\\\end{tabular}
\end{table}


\item[aktueller Dampfdruck]
\begin{equation}
\label{eq:ea}
e_a=\frac{RH_{mittel}}{100}*e_s
\end{equation}
\begin{table}[H]
\centering
\begin{tabular}{ll}
$\mathrm{e_a}$ & aktueller Dampfdruck $\mathrm{[kPa]}$\\
$\mathrm{e_s}$ & mittlerer Sättigungsdampfdruck $\mathrm{[kPa]}$\\
$\mathrm{RH_{mittel}}$ & mittlere relative Luftfeuchtigkeit [\%]\\
\end{tabular}
\end{table}

\end{description}


\section{Sensitivitätsanalyse}
\label{sec:sensitivitaet}

\begin{figure}[H]
\centering
\includegraphics[width=0.8\textwidth]{figures/sensitivitaet.jpg}
\caption{tabellarische Berechnung der Sensitivität von der Penman-Monteith-, Turc- und Ivanov-Methode. Die markierten Felder wurden  verändert.}
\label{fig:lysimeter_ART}
\end{figure}

\end{appendix}


% für Literaturverzeichnis
\bibliographystyle{natdin}
\bibliography{experiment4}


\end{document}

$ latex myarticle
$ bibtex myarticle
$ latex myarticle
$ latex myarticle
